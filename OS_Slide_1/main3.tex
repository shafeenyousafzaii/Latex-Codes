\documentclass{article}
\usepackage{amsmath}
\usepackage{enumitem}
\usepackage{hyperref}

\title{Summary of Chapter 3: Processes}
\author{}
\date{}

\begin{document}
\maketitle

\section{Chapter 3: Processes}
This chapter discusses the concept of processes, process scheduling, interprocess communication (IPC), and client-server communication systems.

\section{Objectives}
The key goals are:
\begin{itemize}
    \item To introduce the concept of a process—a program in execution.
    \item To describe features of processes, including scheduling, creation, termination, and communication.
    \item To explore interprocess communication using shared memory and message passing.
    \item To describe communication in client-server systems.
\end{itemize}

\section{Process Concept}
A process is a program in execution. It includes:
\begin{itemize}
    \item \textbf{Text Section}: Executable code.
    \item \textbf{Program Counter}: Current activity and processor registers.
    \item \textbf{Stack}: Temporary data (e.g., function parameters, return addresses).
    \item \textbf{Data Section}: Global variables.
    \item \textbf{Heap}: Dynamic memory.
\end{itemize}

Processes differ from programs, as one program can be executed as multiple processes.

\section{Process State}
A process can be in one of several states:
\begin{itemize}
    \item \textbf{New}: The process is being created.
    \item \textbf{Running}: Instructions are being executed.
    \item \textbf{Waiting}: The process is waiting for an event.
    \item \textbf{Ready}: The process is ready to be assigned to the CPU.
\end{itemize}

\section{Process Control Block (PCB)}
The PCB contains information associated with a specific process, including:
\begin{itemize}
    \item Process state.
    \item Program counter.
    \item CPU registers.
    \item Scheduling and memory management info.
    \item I/O status.
\end{itemize}

\section{Process Scheduling}
The process scheduler selects which process should execute next:
\begin{itemize}
    \item \textbf{Job Queue}: All processes in the system.
    \item \textbf{Ready Queue}: Processes ready to execute.
    \item \textbf{Device Queue}: Processes waiting for I/O.
\end{itemize}

Schedulers:
\begin{itemize}
    \item \textbf{Short-term scheduler (CPU scheduler)} selects processes for execution.
    \item \textbf{Long-term scheduler (job scheduler)} controls the degree of multiprogramming by determining which processes are brought into the ready queue.
\end{itemize}

\section{Context Switch}
When switching processes, the system must save the current process state and load the saved state of the new process. This is called a \textbf{context switch}, and it introduces overhead since the CPU isn't doing useful work during the switch.

\section{Operations on Processes}
The OS must support various operations on processes, such as:
\begin{itemize}
    \item \textbf{Process Creation}: Parent processes create children, forming a tree of processes. UNIX uses the \texttt{fork()} system call.
    \item \textbf{Process Termination}: Processes terminate by invoking \texttt{exit()}, and resources are deallocated. Parent processes may wait for children to terminate.
\end{itemize}

\section{Interprocess Communication (IPC)}
Processes may need to communicate with each other:
\begin{itemize}
    \item \textbf{Cooperating processes} share data and resources.
    \item IPC models include:
    \begin{itemize}
        \item \textbf{Shared Memory}: Processes share a portion of memory.
        \item \textbf{Message Passing}: Processes send and receive messages.
    \end{itemize}
\end{itemize}

\section{Message Passing}
Messages are passed between processes via communication links:
\begin{itemize}
    \item \textbf{Direct Communication}: Processes must explicitly name each other.
    \item \textbf{Indirect Communication}: Processes use mailboxes (ports) to send and receive messages.
    \item \textbf{Synchronization}: Message passing can be blocking (synchronous) or non-blocking (asynchronous).
\end{itemize}

\section{Buffering}
Messages are stored in queues:
\begin{itemize}
    \item \textbf{Zero Capacity}: Sender must wait for the receiver.
    \item \textbf{Bounded Capacity}: Finite queue size, sender waits if the queue is full.
    \item \textbf{Unbounded Capacity}: Infinite queue size, sender never waits.
\end{itemize}

\section{Communication in Client-Server Systems}
\begin{itemize}
    \item \textbf{Sockets}: Endpoint for communication, identified by IP address and port number.
    \item \textbf{Pipes}: Communication channels between processes. Named pipes allow communication between unrelated processes, whereas ordinary pipes require a parent-child relationship.
\end{itemize}

This chapter lays the foundation for understanding process management and communication, essential components of modern operating systems.

\end{document}
