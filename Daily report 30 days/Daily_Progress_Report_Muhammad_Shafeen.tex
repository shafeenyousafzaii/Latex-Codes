\documentclass{report}

\usepackage{hyperref}
\usepackage{graphicx}

\title{Daily Progress Report}
\author{Muhammad Shafeen supervised by Zeeshan Khalid }
\date{July 11, 2024 – August 11, 2024}

\begin{document}

\maketitle

\tableofcontents

\chapter{Introduction}
This report provides a detailed account of the work carried out by Muhammad Shafeen at Innotech Solutions under the supervision of Zeeshan in the field of Artificial Intelligence from July 11, 2024, to August 11, 2024. The projects undertaken during this period involved various aspects of AI, including object detection, dataset creation, and LLM optimization.

\chapter{Daily Reports}

\section{July 11 - July 17: Offline Weather Project (Finding Dataset)}[H]

\subsection*{July 11}
Began searching for the Kargil dataset required for training the offline weather prediction model. The initial challenges involved limited availability and high costs of data sources.

\subsection*{July 12}
Developed and executed a Python script to scrape the required data for Kargil, addressing issues with access and scraping efficiency.

\subsection*{July 13}
Completed the data scraping process, gathered data from 1970 to 2024. Began preprocessing the dataset for model training.

\subsection*{July 14}
Conducted a thorough analysis of the dataset, focusing on completeness and data quality. Identified gaps and inconsistencies.

\subsection*{July 15}
Filled gaps in the dataset by sourcing alternative data and refining the preprocessing pipeline.

\subsection*{July 16}
Initiated training of the offline weather prediction model using the gathered Kargil dataset. Early results showed promise.

\subsection*{July 17}
Evaluated the model’s performance against test data, documenting the initial findings and preparing for further iterations.

\section{July 18 - July 22: MediaPipe LLM Project}[H]

\subsection*{July 18}
Installed and configured the MediaPipe framework. Set up the Android Studio environment for LLM deployment.

\subsection*{July 19}
Trained a custom Language Model (LLM) on Google Colab using the Gamma model as a base, ensuring that it is optimized for mobile use.

\subsection*{July 20}
Integrated the trained model into an Android application and conducted preliminary tests.

\subsection*{July 21}
Identified performance issues on older devices during testing. Began working on optimizations to reduce the model size and enhance performance.

\subsection*{July 22}
Continued optimization efforts, focusing on reducing latency and improving compatibility across a broader range of mobile devices.

\section{July 23 - July 27: YOLO Object Detection}[H]

\subsection*{July 23}
Set up the YOLOv8 object detection model, laying the groundwork for training on custom datasets.

\subsection*{July 24}
Collected and labeled data for training YOLO on specific objects of interest, preparing the dataset for model input.

\subsection*{July 25}
Trained the YOLOv8 model on the labeled dataset. Initial results showed the model’s potential, with some areas needing further refinement.

\subsection*{July 26}
Tested the model in a real-time detection scenario. Noted areas where the model could be improved, particularly in reducing false positives.

\subsection*{July 27}
Fine-tuned the YOLO model parameters to enhance detection accuracy and reduce false positives, preparing the model for deployment.

\section{July 28 - August 1: Finding Jet Dataset}[H]

\subsection*{July 28}
Began the search for a comprehensive jet dataset to be used in YOLO-based detection tasks. Faced initial challenges in finding freely available data.

\subsection*{July 29}
Identified several potential data sources, but encountered issues with data accessibility and licensing.

\subsection*{July 30}
Scraped available jet images and metadata from open-source platforms, ensuring the data was suitable for training purposes.

\subsection*{July 31}
Preprocessed the gathered dataset, ensuring consistency and preparing it for YOLO model integration.

\subsection*{August 1}
Analyzed the dataset’s suitability for training and refined the data preprocessing pipeline, readying it for the next phase.

\section{August 2 - August 6: Face Recognition Project}[H]

\subsection*{August 2}
Set up the initial face recognition model, choosing appropriate architecture and tools for training.

\subsection*{August 3}
Collected a diverse set of face images, ensuring the dataset was representative of different demographics and conditions.

\subsection*{August 4}
Trained the face recognition model using deep learning techniques, focusing on accuracy and robustness.

\subsection*{August 5}
Tested the model on real-world data, analyzing performance in various scenarios and identifying areas for improvement.

\subsection*{August 6}
Worked on enhancing recognition accuracy, addressing challenges related to dataset diversity and model generalization.

\section{August 7 - August 11: YOLO Detection of Jets}[H]

\subsection*{August 7}
Integrated the jet dataset into the YOLO model, beginning the training process for jet detection tasks.

\subsection*{August 8}
Trained the YOLO model specifically on jet detection, focusing on achieving high accuracy with minimal false positives.

\subsection*{August 9}
Evaluated the model’s performance on test data, fine-tuned the model to improve results.

\subsection*{August 10}
Conducted extensive testing to ensure the model’s robustness in various scenarios, preparing for deployment.

\subsection*{August 11}
Finalized the documentation of the project’s results and prepared the model for integration into broader systems.

{Project Summaries}[H]

\section{Androidyolofirebase}[H]
Details on the integration of YOLO models with Firebase for Android applications were explored. The setup was successful, with plans for further enhancements.

\section{MediaPipe LLM}[H]
The project involved setting up and optimizing custom LLMs for mobile devices using MediaPipe. While the initial setup was successful, further optimization is required for better performance on older devices.

\section{Offline Weather Project}[H]
This project focused on gathering datasets for offline weather prediction models. The initial challenges were overcome with automated data scraping techniques.

\section{Yolo Object Detection}[H]
This project involved setting up and training YOLO models for object detection tasks. Further details on optimization and real-time deployment will be included as the project progresses.

\section{Finding Jet Dataset}[H]
The search for a comprehensive jet dataset posed challenges similar to the Kargil dataset. Continued efforts are being made to locate suitable data sources.

\section{Face Recognition}[H]
The face recognition project involved setting up and training models for accurate identification. Challenges with dataset variety were noted, and ongoing efforts are being made to improve accuracy.

\section{Yolo Detection of Jets}[H]
This project combined YOLO models with the jet dataset to create a robust detection system. Initial results are promising, with plans for further development.

\chapter{Conclusion}
This report summarizes the significant progress made in various AI projects at Innotech Solutions. Each project presented unique challenges, and ongoing efforts will focus on optimization and deployment. The next steps involve refining models and expanding the scope of each project.

\end{document}
