\documentclass[a4paper,12pt]{article}
\usepackage{graphicx}
\usepackage{amsmath}
\usepackage{fancyhdr}
\usepackage{geometry}
\usepackage{parskip} % Adds spacing between paragraphs for readability
\usepackage{tcolorbox} % For creating answer boxes

% Page settings
\geometry{margin=1in}
\setlength{\parindent}{0pt} % No indentation
\pagestyle{fancy}
\fancyhf{}
\fancyhead[L]{\includegraphics[width=1.5cm]{logo.png}} % Include university logo
\fancyhead[C]{\textbf{National University of Computer and Emerging Sciences}}

% Adjust header spacing
\setlength{\headheight}{47.02646pt} % Ensure there is enough space for the header
\addtolength{\topmargin}{-27.02646pt} % Compensate by reducing the top margin

% Student information
\newcommand{\studentinfo}{
    \textbf{Student Name:} \underline{\hspace{4cm}} \hspace{0.5cm}
    \textbf{Roll No:} \underline{\hspace{3cm}} \hspace{0.5cm} \\
    \textbf{Section:} \underline{\hspace{2cm}} \\
}

\begin{document}

% Title
\begin{center}
    {\large \textbf{Course: Programming Fundamentals}} \\
    \vspace{0.2cm}
    {\Large \textbf{Quiz 2}} \\
\end{center}

\vspace{0.5cm}

% Student information section
\studentinfo

\vspace{0.5cm}
\textbf{Total Time:} 25 minutes \\
\textbf{Time Distribution:} Q1 - 5 minutes, Q2 (Flowchart) - 5 minutes, Q2 (Code) - 8 minutes, Q3 - 4 minutes, Rechecking - 3 minutes.
\vspace{0.5cm}

% Question 1
\textbf{Q1: Write a C++ program to check whether a triangle is Equilateral, Isosceles, or Scalene.} \hfill \textbf{(5 marks)}  \textbf{(CLO-2)}

 \vspace{0.5cm}
- \textbf{Equilateral Triangle:} A triangle in which all three sides are equal. \\
- \textbf{Isosceles Triangle:} A triangle with two sides of equal length. \\
- \textbf{Scalene Triangle:} A triangle with three unequal sides. \\


\textbf{Write your answer here:}
\begin{tcolorbox}[colframe=black, colback=white, width=\textwidth, height=12cm, valign=center]
\end{tcolorbox}

\vspace{0.5cm}

% Question 2
\textbf{Q2: Weather Prediction - Decision to Play Tennis}  \\
You are provided with the following records of weather conditions and the decision to play tennis:

\begin{center}
\begin{tabular}{|c|c|c|c|}
\hline
\textbf{Temperature (°F)} & \textbf{Humidity (M/H/L)} & \textbf{Windy (Y/N)} & \textbf{Play Tennis (Y/N)} \\
\hline
Above 85 & Medium (M) & No (N) & Yes (Y) \\
75 - 85 & High (H) & Yes (Y) & No (N) \\
Below 75 & Low (L) & No (N) & Yes (Y) \\
\hline
\end{tabular}
\end{center}

\textbf{Task:}
\begin{itemize}
    \item \textbf{a)} Create a flowchart for the above scenario. \hfill \textbf{(5 marks)} \textbf{(CLO-1)}
    \item \textbf{b)} Write a C++ program that uses an if-else structure to predict whether to "Play Tennis" or not based on three input values: temperature (°F), humidity (Medium, High, Low), and whether it's windy (Yes/No). \hfill \textbf{(5 marks)} \textbf{(CLO-2)}
    \begin{itemize}
        \item Accept input for temperature (°F), humidity (M, H, L), and windy (Y/N).
        \item Compare the input with the dataset above.
        \item Output "Y" or "N" based on the conditions provided.
    \end{itemize}
\end{itemize}

\textbf{Example:}
\begin{itemize}
    \item Input: Temperature = 86, Humidity = M, Windy = N
    \item Output: Y
\end{itemize}

\vspace{0.5cm}
\textbf{Write your answer here:}
\begin{tcolorbox}[colframe=black, colback=white, width=\textwidth, height=10cm, valign=center]
\end{tcolorbox}


\vspace{0.5cm}
\textbf{Write your answer here:}
\begin{tcolorbox}[colframe=black, colback=white, width=\textwidth, height=23cm, valign=center]
\end{tcolorbox}
\vspace{0.5cm}
% Question 3
\textbf{Q3: Write the output of the following code:} \hfill \textbf{(5 marks)} \textbf{(CLO-1)}

\begin{verbatim}
#include <iostream>
#include <iomanip>

using namespace std;

int main() {
    int num1 = 5 - (17 * 6);  // 0.5 marks
    int num2 = (5 - 17) * 6;  // 0.5 marks
    int num3 = 19 + (5 / 2) + 4 % 2 - 6 * 3;  // 1 mark

    cout << "num1 = " << num1 << endl;
    cout << "num2 = " << num2 << endl;
    cout << "num3 = " << num3 << endl;

    cout << left << setw(8) << setfill('0') << 256 << endl;  // 1 mark
    cout << left << setfill('*') << setw(8) << num1 << endl; // 1 mark
    cout << right << setw(6) << num2 << endl;  // 1 mark

    return 0;
}
\end{verbatim}

\vspace{0.5cm}
\textbf{Write your answer here:}
\begin{tcolorbox}[colframe=black, colback=white, width=\textwidth, height=6cm, valign=center]
\end{tcolorbox}

\vspace{0.5cm}

\end{document}
