\documentclass[12pt]{article}
\usepackage{graphicx}
\usepackage{geometry}
\usepackage{titlesec}
\usepackage{fancyhdr}

% Geometry settings
\geometry{a4paper, margin=1in}

% Title formatting
\titleformat{\section}{\normalfont\Large\bfseries}{\thesection}{1em}{}

% Header and footer settings
\pagestyle{fancy}
\fancyhf{}
\fancyhead[L]{Physiology-II Project Report}
\fancyhead[R]{\thepage}
\fancyfoot[L]{Submitted to: Ma'am Haleema}
\fancyfoot[R]{Date: 29/05/2024}

% Title page
\title{
    \vspace{-2cm}
    \includegraphics[width=0.3\textwidth]{logo.png}\\ % Include your institution logo if available
    \vspace{1cm}
    \textbf{Physiology-II Project Report:}\\
    \textbf{Working Model of Human Lungs}
}
\author{
    \textbf{Project Members:} \\
    Dua Farooq (009) \\
    Javeria Ahmed (018) \\
    Shaheer
}
\date{29/05/2024}

\begin{document}

\maketitle
\newpage

\tableofcontents
\newpage

\section*{Abstract}
This project involves creating a working model of the human lungs to demonstrate the process of respiration. The model uses simple materials such as cardboard, balloons, a syringe, and household items to illustrate how the lungs inflate and deflate during breathing. This model aims to provide a visual and practical understanding of the respiratory system’s mechanics.

\section{Objective}
To construct a functional model of human lungs that effectively demonstrates the inhalation and exhalation process.

\section{Materials and Tools}
\begin{itemize}
    \item Cardboard
    \item Masking tape
    \item 1 Meter pipe
    \item Hot glue gun
    \item Fevicol (white glue)
    \item Tissue paper
    \item Aluminum foil
    \item 60ml syringe
    \item Rubber band
    \item Acrylic color
    \item Balloon
\end{itemize}

\section{Introduction}
The respiratory system is essential for gas exchange in the human body. The lungs are the primary organs responsible for this process, where oxygen is taken in, and carbon dioxide is expelled. This project aims to create a simple yet effective model to represent the functioning of human lungs, providing a tangible demonstration of how breathing works.

\section{Theory}
The respiratory system consists of the airways, lungs, and respiratory muscles. The primary function is to deliver oxygen to the blood and remove carbon dioxide. The process of breathing involves two main phases:
\begin{itemize}
    \item \textbf{Inhalation (Inspiration)}: The diaphragm contracts and moves downward, enlarging the thoracic cavity and reducing the pressure inside the lungs, causing air to flow in.
    \item \textbf{Exhalation (Expiration)}: The diaphragm relaxes and moves upward, reducing the thoracic cavity's size and increasing the pressure inside the lungs, causing air to be expelled.
\end{itemize}
The lungs are spongy organs located in the chest cavity, protected by the rib cage. Air enters through the trachea, which divides into bronchi and then into smaller bronchioles, eventually reaching the alveoli where gas exchange occurs.

\section{Procedure}

\subsection{Base Construction}
\begin{enumerate}
    \item Cut a rectangular piece of cardboard to serve as the base.
    \item Construct a cardboard frame to hold the pipe vertically in place.
    \item Secure the frame and the pipe to the base using masking tape and hot glue.
\end{enumerate}

\subsection{Trachea and Bronchi}
\begin{enumerate}
    \item Attach the pipe vertically in the center of the base to represent the trachea.
    \item Divide the top end of the pipe into two smaller pipes to simulate the bronchi.
\end{enumerate}

\subsection{Lungs Construction}
\begin{enumerate}
    \item Inflate the balloons slightly and secure the openings with rubber bands.
    \item Attach each balloon to the end of the smaller pipes using hot glue and masking tape.
\end{enumerate}

\subsection{Diaphragm}
\begin{enumerate}
    \item Cut a piece of tissue paper to fit around the bottom of the balloons and the pipe.
    \item Secure the tissue paper with Fevicol to simulate the diaphragm.
    \item Connect the 60ml syringe to the tissue paper diaphragm using hot glue.
\end{enumerate}

\subsection{Finishing Touches}
\begin{enumerate}
    \item Cover the model with aluminum foil for a polished look.
    \item Paint the model using acrylic colors to enhance its appearance.
\end{enumerate}

\section{Working Principle}
The working model demonstrates the basic mechanics of breathing. The diaphragm, represented by tissue paper, moves downwards when the syringe plunger is pulled, creating a vacuum in the chest cavity and causing the balloons (lungs) to inflate (inhale). When the syringe plunger is pushed, the diaphragm moves upwards, increasing the pressure in the chest cavity and causing the balloons to deflate (exhale). This mimics the natural process of inhalation and exhalation in human lungs.

\section{Conclusion}
The constructed model successfully illustrates the process of respiration, showing how the diaphragm’s movement affects the lungs' ability to take in and expel air. This hands-on project enhances the understanding of the respiratory system’s functionality, making it an effective educational tool.

\section{References}
\begin{itemize}
    \item Basic human anatomy textbooks
    \item Online resources on DIY respiratory models
\end{itemize}

\end{document}
