\documentclass{article}
\usepackage{graphicx}

\title{Solar System Explorer: Knowledge Representation System Proposal}
\author{}
\date{Fall 2024}

\begin{document}  % This adds 1cm of vertical space before the title
\maketitle


\section{Introduction}
The purpose of this project is to develop a knowledge representation system for celestial bodies within our solar system, including planets, moons, and asteroids. The system will gather data either through web scraping or a suitable dataset, which has yet to be finalized, to build a query-able, structured knowledge base.

\section{Objectives}
The main objective is to represent solar system data in a formalized manner that allows for easy querying and reasoning. This project will:
\begin{itemize}
    \item Collect and preprocess data via web scraping or an external dataset.
    \item Develop an ontology that models the relationships between celestial bodies.
    \item Implement a query system for extracting information from the knowledge base.
\end{itemize}

\section{Proposed Methodology}
The project will be divided into four main phases:
\begin{enumerate}
    \item \textbf{Data Collection and Preprocessing}: Gather and clean data on planets, moons, and asteroids using web scraping or from a suitable dataset.
    \item \textbf{Ontology Development}: Use ontology tools (such as Protégé) to model relationships and attributes between celestial bodies.
    \item \textbf{Query Implementation}: Implement a SPARQL-based query system to enable interactive queries on the dataset.
    \item \textbf{Testing and Evaluation}: Test the system using predefined queries and evaluate its accuracy and performance.
\end{enumerate}

% \section{Timeline}
% \begin{itemize}
%     \item \textbf{Week 1}: Data Collection and Preprocessing.
%     \item \textbf{Week 2}: Ontology Development and Relationship Modeling.
%     \item \textbf{Week 3}: Query Implementation.
%     \item \textbf{Week 4}: Testing, Evaluation, and Documentation.
% \end{itemize}

\section{Conclusion}
This project will create a functional, query-able knowledge representation system for the solar system. The system will enable users to retrieve detailed information about planets, moons, and other celestial bodies, facilitating further research and exploration.

\end{document}
