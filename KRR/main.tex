\documentclass{article}
\usepackage{graphicx}

\title{Solar System Explorer: Knowledge Representation System Proposal}
\author{}
\date{Fall 2024}

\begin{document}

\maketitle

\section{Introduction}
The purpose of this project is to develop a knowledge representation system for celestial bodies within our solar system, including planets, moons, and asteroids. The system will utilize data from the NASA Open Data Portal (Solar System section) to build a query-able, structured knowledge base.

\section{Objectives}
The main objective is to represent solar system data in a formalized manner that allows for easy querying and reasoning. This project will:
\begin{itemize}
    \item Collect and preprocess data from the NASA Open Data Portal.
    \item Develop an ontology that models the relationships between celestial bodies.
    \item Implement a query system for extracting information from the knowledge base.
\end{itemize}

\section{Proposed Methodology}
The project will be divided into four main phases:
\begin{enumerate}
    \item \textbf{Data Collection and Preprocessing}: Acquire and clean data on planets, moons, and asteroids from NASA’s Open Data Portal.
    \item \textbf{Ontology Development}: Use ontology tools (such as Protégé) to model relationships and attributes between celestial bodies.
    \item \textbf{Query Implementation}: Implement a SPARQL-based query system to enable interactive queries on the dataset.
    \item \textbf{Testing and Evaluation}: Test the system using predefined queries and evaluate its accuracy and performance.
\end{enumerate}

\section{Timeline}
This is the propose timeline for a project during implementation.Can be extended.
\begin{itemize}
    \item \textbf{1 week}: Data Collection and Preprocessing.
    \item \textbf{2 week}: Ontology Development and Relationship Modeling.
    \item \textbf{3 week}: Query Implementation.
    \item \textbf{4 week}: Testing, Evaluation, and Documentation.
\end{itemize}

\section{Conclusion}
This project will create a functional, query-able knowledge representation system for the solar system. The system will enable users to retrieve detailed information about planets, moons, and other celestial bodies, facilitating further research and exploration.

\end{document}
