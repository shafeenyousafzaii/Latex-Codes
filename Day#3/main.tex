\documentclass{article}
\usepackage{graphicx}
\usepackage{hyperref}
\usepackage{amsmath}
\usepackage{geometry}
\geometry{a4paper, margin=1in}

\title{Day 2 : Approaches for Future Prediction: A Comparative Study}
\author{Muhammad Shafeen}
\date{July 16, 2024}

\begin{document}

\maketitle

\section{Introduction}
The task for Day 3 was to explore different approaches for future prediction, focusing on three distinct methods. This report provides an overview of each method, supported by corresponding GitHub repositories, to offer a broader understanding of their implementation and workings.

\section{Approach 1: Weather Forecasting Using Machine Learning}
\subsection{Overview}
The first approach involves using machine learning techniques for weather forecasting. This method leverages historical weather data to predict future conditions.

\subsection{Repository}
The implementation can be found in the following GitHub repository: \href{https://github.com/Hrishikesh332/Weather-Forecasting-Using-ML}{Weather-Forecasting-Using-ML}.

\subsection{Details}
The repository provides a comprehensive solution for weather forecasting using various machine learning algorithms. It includes data preprocessing, model training, and evaluation processes. The algorithms used include linear regression, decision trees, and random forests, among others.

\begin{figure}[h]
\centering
\caption{Example output from Weather Forecasting Using ML}
\end{figure}

\section{Approach 2: LSTM for Climatological Time Series}
\subsection{Overview}
The second approach utilizes Long Short-Term Memory (LSTM) networks, a type of recurrent neural network (RNN), for time series prediction, specifically for climatological data.

\subsection{Repository}
The implementation details are available in the GitHub repository: \href{https://github.com/danielefranceschi/lstm-climatological-time-series?tab=readme-ov-file}{LSTM Climatological Time Series}.

\subsection{Details}
LSTM networks are well-suited for time series prediction due to their ability to capture long-term dependencies in sequential data. This repository demonstrates how LSTM networks can be trained on climatological data to make future predictions, including data preprocessing, model architecture, training, and evaluation.

\begin{figure}[h]
\centering
\caption{Example output from LSTM Climatological Time Series}
\end{figure}

\section{Approach 3: ARIMA/SARIMA for Time Series Forecasting}
\subsection{Overview}
The third approach explores the use of ARIMA (AutoRegressive Integrated Moving Average) and SARIMA (Seasonal ARIMA) models for time series forecasting.

\subsection{Repository}
The implementation can be accessed through the GitHub repository: \href{https://github.com/javaidiqbal11/Time-Series-Forecasting-using-ARIMA-SARIMA}{Time-Series-Forecasting-using-ARIMA-SARIMA}.

\subsection{Details}
ARIMA and SARIMA models are powerful statistical methods for analyzing and forecasting time series data. The repository provides a detailed implementation of these models, including data preprocessing, parameter selection, model training, and performance evaluation.

\begin{figure}[h]
\centering
\caption{Example output from ARIMA/SARIMA Time Series Forecasting}
\end{figure}

\section{Conclusion}
This report has outlined three different approaches for future prediction: machine learning for weather forecasting, LSTM networks for climatological time series, and ARIMA/SARIMA models for time series forecasting. Each method has its unique advantages and is suited for specific types of data and prediction tasks. The provided GitHub repositories offer detailed implementations and serve as valuable resources for understanding and applying these techniques.

\end{document}
