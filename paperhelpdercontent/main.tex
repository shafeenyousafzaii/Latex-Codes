\documentclass{article}
\usepackage{listings}
\usepackage{color}

\definecolor{lightgray}{gray}{0.9}

\lstset{
  backgroundcolor=\color{lightgray},
  basicstyle=\ttfamily,
  keywordstyle=\color{blue},
  commentstyle=\color{green},
  stringstyle=\color{red},
  showstringspaces=false,
  frame=single,
  breaklines=true
}

\begin{document}

\title{SQL Guide for Teaching and Exam Preparation}
\author{Database Expert}
\date{\today}
\maketitle

\tableofcontents
\newpage

\section{Creating Tables with Primary Keys and Foreign Keys}

\subsection{Format}
\begin{lstlisting}[language=SQL]
CREATE TABLE table_name (
    column1 datatype PRIMARY KEY,
    column2 datatype,
    column3 datatype,
    ...
    FOREIGN KEY (column_name) REFERENCES parent_table (column_name)
);
\end{lstlisting}

\subsection{Example}
\begin{lstlisting}[language=SQL]
CREATE TABLE departments (
    department_id INT PRIMARY KEY,
    department_name VARCHAR(100)
);

CREATE TABLE employees (
    employee_id INT PRIMARY KEY,
    employee_name VARCHAR(100),
    department_id INT,
    FOREIGN KEY (department_id) REFERENCES departments(department_id)
);
\end{lstlisting}

\newpage
\section{Joins}

\subsection{Inner Join}
\begin{lstlisting}[language=SQL]
SELECT columns
FROM table1
INNER JOIN table2
ON table1.column = table2.column;
\end{lstlisting}

\subsection{Example}
\begin{lstlisting}[language=SQL]
SELECT employees.employee_name, departments.department_name
FROM employees
INNER JOIN departments
ON employees.department_id = departments.department_id;
\end{lstlisting}

\subsection{Left Join}
\begin{lstlisting}[language=SQL]
SELECT columns
FROM table1
LEFT JOIN table2
ON table1.column = table2.column;
\end{lstlisting}

\subsection{Example}
\begin{lstlisting}[language=SQL]
SELECT employees.employee_name, departments.department_name
FROM employees
LEFT JOIN departments
ON employees.department_id = departments.department_id;
\end{lstlisting}

\subsection{Right Join}
\begin{lstlisting}[language=SQL]
SELECT columns
FROM table1
RIGHT JOIN table2
ON table1.column = table2.column;
\end{lstlisting}

\subsection{Example}
\begin{lstlisting}[language=SQL]
SELECT employees.employee_name, departments.department_name
FROM employees
RIGHT JOIN departments
ON employees.department_id = departments.department_id;
\end{lstlisting}

\subsection{Full Outer Join}
\begin{lstlisting}[language=SQL]
SELECT columns
FROM table1
FULL OUTER JOIN table2
ON table1.column = table2.column;
\end{lstlisting}

\subsection{Example}
\begin{lstlisting}[language=SQL]
SELECT employees.employee_name, departments.department_name
FROM employees
FULL OUTER JOIN departments
ON employees.department_id = departments.department_id;
\end{lstlisting}

\newpage
\section{Subqueries}

\subsection{Format}
\begin{lstlisting}[language=SQL]
SELECT columns
FROM table
WHERE column = (SELECT column FROM table WHERE condition);
\end{lstlisting}

\subsection{Example}
\begin{lstlisting}[language=SQL]
SELECT employee_name
FROM employees
WHERE department_id = (SELECT department_id FROM departments WHERE department_name = 'Sales');
\end{lstlisting}

\newpage
\section{Creating Views and SQL Functions}

\subsection{Views}

\subsubsection{Format}
\begin{lstlisting}[language=SQL]
CREATE VIEW view_name AS
SELECT columns
FROM table
WHERE condition;
\end{lstlisting}

\subsubsection{Example}
\begin{lstlisting}[language=SQL]
CREATE VIEW sales_employees AS
SELECT employee_name, department_id
FROM employees
WHERE department_id = (SELECT department_id FROM departments WHERE department_name = 'Sales');
\end{lstlisting}

\subsection{Functions}

\subsubsection{Format}
\begin{lstlisting}[language=SQL]
CREATE FUNCTION function_name (parameters)
RETURNS return_datatype
AS
BEGIN
    -- SQL statements
    RETURN value;
END;
\end{lstlisting}

\subsubsection{Example}
\begin{lstlisting}[language=SQL]
CREATE FUNCTION get_department_name (dept_id INT)
RETURNS VARCHAR(100)
AS
BEGIN
    DECLARE dept_name VARCHAR(100);
    SELECT department_name INTO dept_name
    FROM departments
    WHERE department_id = dept_id;
    RETURN dept_name;
END;
\end{lstlisting}

\newpage
\section{Creating Triggers}

\subsection{Format}
\begin{lstlisting}[language=SQL]
CREATE TRIGGER trigger_name
AFTER | BEFORE INSERT | UPDATE | DELETE
ON table_name
FOR EACH ROW
BEGIN
    -- SQL statements
END;
\end{lstlisting}

\subsection{Example}
\begin{lstlisting}[language=SQL]
CREATE TRIGGER after_employee_insert
AFTER INSERT
ON employees
FOR EACH ROW
BEGIN
    INSERT INTO audit_log (action, timestamp)
    VALUES ('New employee added', NOW());
END;
\end{lstlisting}

\newpage
\section{Creating Procedures}

\subsection{Format}
\begin{lstlisting}[language=SQL]
CREATE PROCEDURE procedure_name (parameters)
BEGIN
    -- SQL statements
END;
\end{lstlisting}

\subsection{Example}
\begin{lstlisting}[language=SQL]
CREATE PROCEDURE add_employee (emp_name VARCHAR(100), dept_id INT)
BEGIN
    INSERT INTO employees (employee_name, department_id)
    VALUES (emp_name, dept_id);
END;
\end{lstlisting}

\newpage
\section{DCL (Data Control Language) and TCL (Transaction Control Language)}

\subsection{DCL}

\subsubsection{GRANT}
\begin{lstlisting}[language=SQL]
GRANT privilege ON object TO user;
\end{lstlisting}

\subsubsection{Example}
\begin{lstlisting}[language=SQL]
GRANT SELECT ON employees TO 'user1';
\end{lstlisting}

\subsubsection{REVOKE}
\begin{lstlisting}[language=SQL]
REVOKE privilege ON object FROM user;
\end{lstlisting}

\subsubsection{Example}
\begin{lstlisting}[language=SQL]
REVOKE SELECT ON employees FROM 'user1';
\end{lstlisting}

\subsection{TCL}

\subsubsection{COMMIT}
\begin{lstlisting}[language=SQL]
COMMIT;
\end{lstlisting}

\subsubsection{Example}
\begin{lstlisting}[language=SQL]
INSERT INTO employees (employee_name, department_id) VALUES ('John Doe', 1);
COMMIT;
\end{lstlisting}

\subsubsection{ROLLBACK}
\begin{lstlisting}[language=SQL]
ROLLBACK;
\end{lstlisting}

\subsubsection{Example}
\begin{lstlisting}[language=SQL]
INSERT INTO employees (employee_name, department_id) VALUES ('John Doe', 1);
ROLLBACK;
\end{lstlisting}

\subsubsection{SAVEPOINT}
\begin{lstlisting}[language=SQL]
SAVEPOINT savepoint_name;
\end{lstlisting}

\subsubsection{Example}
\begin{lstlisting}[language=SQL]
INSERT INTO employees (employee_name, department_id) VALUES ('John Doe', 1);
SAVEPOINT savepoint1;
INSERT INTO employees (employee_name, department_id) VALUES ('Jane Smith', 2);
ROLLBACK TO savepoint1;
\end{lstlisting}

\end{document}
